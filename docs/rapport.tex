\documentclass{article}

\usepackage[utf8]{inputenc}
\usepackage[top=1.25in, bottom=1.25in, left=0.9in, right=0.9in]{geometry}
\usepackage[T1]{fontenc}
\usepackage[frenchb]{babel}
\usepackage{array}
\usepackage{fancyhdr}
\usepackage{amssymb}
\usepackage[final]{pdfpages}
\usepackage{biblatex}

\pagestyle{fancy}
\lhead{Rapport TP3}
\rhead{Minesweeper}

% pour éviter d'avoir à faire des \noindent partout!

\title{%
\Large{Université du Québec à Montréal}\\
\vspace{2.5cm}
\Huge{Minesweeper}\\
\vspace{3cm}
\Large{Travail présenté à \\M. Eric Beaudry} \\
\vspace{2cm}
\Large{Dans le cadre du cours \\INF4230-10 – Intelligence Artificielle} \\
\vspace{1cm}
\author{Martin Bouchard, BOUM15078700\\Frédéric Vachon, VACF30098405\\Louis-Bertrand Varin,
VARL23089000\\Geneviève Lalonde, LALG08568204\\Nilovna Bascunan-Vasquez, BASN22518900}
\date{\vspace{0.5cm} 15 décembre 2014}
\vfill
}

\begin{document}
\maketitle

\thispagestyle{empty}
\clearpage

\openup .5em

\section{Introduction}
Dans le cadre du troisième TP en Intelligence Artificielle, nous proposons de créer un joueur
artificiel pour le jeu démineur (Minesweeper). Afin d’augmenter la complexité du projet, nous avons
décidé d’implémenter différents algorithmes pour résoudre la grille de Démineur. L’utilisateur sera
en mesure de choisir l’algorithme qu’il désire observer. Les algorithmes utilisés par le joueur
artificiel seront: CSP (Constraint Satisfaction Problem) avec et sans l’heuristique MCV (Most
Constrained Value) pour le choix de la prochaine case, un algorithme de raisonnement probabiliste
avec graphe, le même raisonnement probabiliste avec une heuristique tenant compte du nombre de
drapeaux restants et un simple joueur artificiel qui emploie la technique primitive de sélection
aléatoire.

\section{Problématique et pertinence des techniques employées}
Le jeu démineur appartient à la catégorie des problèmes NP-complets puisqu’il n’est pas toujours
possible de trouver une solution en temps polynomial. Il n’y a pas de méthode connue qui garantit
une meilleure solution que la simple recherche par force brute. S’il y avait un algorithme efficace
connu, il pourrait résoudre tous les problèmes NP-complets de ce type et serait grandement apprécié
par la communauté de mathématiciens (et par notre équipe qui pourrait être millionnaire grâce à ce
concours). Ainsi, lorsque notre joueur artificiel tentera de résoudre une grille, aucun des
algorithmes implémentés ne garantira qu’il ne tombera pas sur une mine. Il s’agit d’un problème
combinatoire souvent intraitable (un problème de prise de décision), comme le démontre le morceau de
grille ci-dessous:

Les cases bleues représentent des cases non découvertes, celles avec un chiffre indiquent le nombre
de mines auxquelles elles touchent (soit par la case immédiatement au-dessus, en-dessous, à droite,
à gauche ou par une des diagonales). Il faut donc trouver quelles sont les cases parmi les bleues
qui contiennent des mines. Ici, il y a 20 cases mystères et 12 cases qui touchent chacune à 2 mines.
En résolvant cette grille, nous savons qu’elle contient 8 mines. Dans une approche naïve, où le
joueur artificiel tenterait toutes les combinaisons possibles à la recherche de la meilleure (celle
brisant le moins de contraintes) pour une grille non découverte de 6 x 6 comme celle ci-haut (donc
36 cases), le nombre de combinaisons possibles de 8 mines serait de l’ordre de:
36!/(8!(36-8)!) = 30 260 340, ce qui correspond au nombre de combinaisons d’une grille plus petite
que le minimum permis dans la version Windows du jeu.

La résolution d’une grille de démineur peut être décomposée à la simple question de savoir quelles
cases sont certaines (d’être minées ou non), et lesquelles sont incertaines. Un bon joueur
artificiel de démineur ne jouera jamais une case incertaine si des cases certaines n’ont pas été
jouées (nommées cases triviales dans nos analyses). 
Il s’agit d’un jeu statique à agent unique partiellement observable, dans un environnement discret
(puisqu’une grille bien délimitée est utilisée). Il s’agit aussi d’un jeu déterministe dans la
mesure où l’action effectuée par le joueur détermine entièrement l’état suivant de la grille, mais
il peut être vu comme non déterministe puisque le joueur ne connaît pas le résultat de l’action
qu’il pose, dû au manque d’observabilité de la grille.
Ce jeu est séquentiel, puisque chaque coup permet de découvrir une partie de la grille et restreint
potentiellement le domaine des variables du prochain coup. Il peut aussi être perçu comme non
séquentiel, puisque la séquence dans laquelle les cases sont cliquées n’a pas d’importance.

Il y a plusieurs algorithmes permettant de résoudre une grille de démineur. Nous avons implémenté
ceux mentionnés plus haut pour les raisons suivantes:
CSP : Les cases du jeu démineur peuvent être représentées comme des variables pouvant adopter
plusieurs valeurs (minée, sans mine, vide, non-découverte) et sont gérées par des contraintes (e.g
deux cases touchant à une case ayant le chiffre 1 ne peuvent pas être toutes les deux minées) et
ainsi bien se prêter à la résolution par CSP.
Heuristique MCV (Most Constrained Value) pour le choix de la prochaine case: notre joueur assumera
que chaque case non-découverte est minée et l’étiquetera d’un drapeau. Par la suite, il vérifiera si
les contraintes sont satisfaites. Lorsqu’il trouvera qu’une contrainte est insatisfaite (car il y
aurait plusieurs drapeaux consécutifs qui totaliseraient plus de mines possibles que les chiffres
correspondants), il utilisera l’heuristique MCV pour retirer des drapeaux:
 la première case est assumée minée (on y pose un drapeau)
 la deuxième case est assumée minée (on y pose aussi un drapeau).
 Contrainte violée: le premier chiffre 1 (celui à gauche) touche à deux mines. Il faut donc enlever
 la valeur “mine” à la première variable (la case à gauche).
 le drapeau de la première case est donc retiré, car il s’agit de la case avec le plus petit domaine
 de valeurs (MCV).
 Raisonnement probabiliste avec graphe: Une autre façon de percevoir une grille de démineur est
 d’imaginer la probabilité qu’il y ait une mine dans chaque case inconnue, en se basant sur les
 chiffres qui sont visibles. 
 Par exemple, dans la configuration :
 , il y a 100\% de chance qu’une mine se trouve dans la case cachée.
 Tandis que dans la configuration :
 , chaque case cachée a une chance sur deux d’avoir la mine. Bien qu’un avide joueur de démineur
 saurait que la mine se situe dans la case du bas, car sinon elle violerait une contrainte, le
 raisonnement probabiliste ne tient pas compte des contraintes.

\section{Résultats}
Afin de nous permettre d’évaluer la performance des techniques précédemment mentionnées, nous avons
soumis nos grilles de démineur àsoumis cent fois, une série d’évaluations sur les différents
algorithmes. Ces critères sont: le nombre de défaites et victoires, Chaque critère a été évalué 100
fois 

Afin de nous permettre d’évaluer la performance des techniques précédemment mentionnées, nous avons
déterminé une série de tests pertinents nous permettant d’extraire des statistiques concernant nos
algorithmes. Ces tests sont:
-bla bla
Chacun d’entre eux ont été répétés sur 5 tailles de grilles différentes. et chaque taille de grille
est évaluée 100 fois

\section{Répartition des tâches}
Martin: Interface et implémentation de l’algorithme CSP, avec et sans MCV, ainsi que d’un joueur
artificiel aléatoire pour tester l’interface, heuristique du nombre de drapeaux restants. 
Frédéric et Louis: Raisonnement probabiliste avec graphe, heuristiques pour des cases aux mêmes
probabilités.
Nilovna et Geneviève: Documentation (lisez-moi, rapport), présentation, revampage de l’interface,
mises à l’essai, statistiques, nettoyage du code et remise.


\section{Conclusion}

\end{document}

